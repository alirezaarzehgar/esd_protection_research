\documentclass[a4paper]{article}

\makeindex
\usepackage{url}
\usepackage[T1]{fontenc}
\usepackage[utf8]{inputenc}
\usepackage[colorlinks=true,urlcolor=blue,linkcolor=blue,breaklinks=true]{hyperref}
\usepackage{graphicx}
\usepackage{geometry}

\geometry{a4paper, margin=1.7cm}

\usepackage{xepersian}
\settextfont{IRNazanin}[
	Extension=.ttf,
	BoldFont=*Bold,
	ItalicFont=*Italic,
	BoldItalicFont=*Bold,
	Path=fonts/
]

\begin{document}

\begin{center}
{\includegraphics[width=1cm]{logo.png}}\\[0.5cm]
گروه کامپیوتر دانشگاه آزاد مشهد\\[0.4cm]
{\LARGE \textbf{مقدمه ای بر حفاظت \lr{ESD}}}\\[0.4cm]
با نظارت: جناب آقای دکتر پژمان خورشید\\[0.4cm]
دانشجو: علیرضا ارزه گر\\[0.4cm]
\today\\[1cm]
\end{center}

حفاظت \lr{ESD}
\LTRfootnote{Electrostatic Discharge Protection}
فرایندی است که در آن از ابزارهای خاصی مانند تجهیزات پزشکی، تجهیزات شبکه و سرورها، آزمایشگاه‌ها،
و هرکجا که تجهیزات الکترونیکی حساس وجود دارند در برابر پدیده \lr{ESD}
\LTRfootnote{Electrostatic Discharge}
محافظت می‌شود.

در دنیای الکترونیک قطعاتی وجود دارند که نزدیک شدن اجسام باردار به آنها می‌تواند باعث بروز اختلال و ایجاد
آسیب شود. یکی از این قطعات \lr{MOSFET} است که بسیار در بردهای الکترونیکی و تجهیزات مختلف کاربرد دارد.

\lr{ESD}
حرکت سریع بارهای الکتریکی از یک شی به شی دیگر است که به آن تخلیه بار الکترواستاتیک نیز می‌گویند.
بار الکترواستاتیک زمانی تولید می‌شود که دو جسم باهم در تماس قرار گیرند و پس از انتقال بار از هم جدا شوند.
در این حالت دیگر این دو جسم خنثی نبوده و هردو دارای بار خواهند بود و هرکدام میدان الکتریکی خواهند داشت.
اجسام باردار درصورتی که نزدیک جسم دیگری شوند بر اساس قوانین کولن می‌توانند به آن نیرو وارد کنند
و القای الکتریکی اتفاق بیوفتد. زمانی که یک جسم خنثی در مجاورت میدان الکتریکی قرار گیرد،
باعث می‌شود در ناحیه ای بار منفی و مقابل آن بار مثبت جمع شود. علاوه بر این الکترون ها در جسم باردار رسانا
در سطح جسم توزیع خواهند شد. مجموع این اتفاقات تخلیه بار الکترواستاتیک را رقم می‌زنند و الکترون‌ها از جسم باردار به
زمین یا جسم مذکور بر اساس قانون کولن شتاب می‌گیرند و پدیده \lr{ESD} اتفاق می‌افتد.
طبق فرایند گفته شده اجسام زیادی در طول روز می‌توانند حاوی بار الکترواستاتیک شوند؛ برای نمونه بدن انسان با راه رفتن بر روی فرش
باردار خواهد شد.

تجهیزات الکترونیکی و خصوصا در این تحقیق تجهیزات موجود در مراکز داده
\LTRfootnote{Data centers}
شامل سوئیچ‌ها، روترها، فایروال‌ها،یو‌پی‌اس‌ها و هر تجهیزی که دارای قطعات الکترونیکی است باید در مقابل \lr{ESD} محافظت شود.
مکان‌هایی که شامل تجهیزاتی هستند که باید در برابر \lr{ESD} محافظت‌شوند، \lr{EPA}
\LTRfootnote{ESD Protected Area}
نام دارند. از مصادیق دیگر \lr{EPA} ها می‌توان به خطوط مونتاژ دستگاه‌های الکترونیکی، آزمایشگاه‌ها و ایستگاه‌های کنترل و تضمین کیفیت،
 آزمایشگاه‌های تحقیق و توسعه در زمینه الکترونیک و اتاق عمل در بیمارستان اشاره کرد. هرکجا که قطعات الکترونیکی وجود دارند و به دلایلی همچون
حساس بودن و اهمیت سالم ماندن تجهیزات، نیاز به حفاظت \lr{ESD} وجود دارد \lr{EPA} نام خواهد گرفت.

در \lr{EPA}ها تکنیک‌های زیادی جهت محافظت \lr{ESD} به‌کار برده می‌شوند. در ابتدا باید منطقه حافظتی را مشخص کنیم.
در مناطق مشخص شده فقط افرادی می‌توانند حضور داشته باشند که از تجهیزات مناسب و امن استفاده می‌کنند. برای امن سازی \lr{EPA}
اقدامات زیر را انجام می‌دهیم:
\begin{itemize}
\item
در تمام مناطق امن \lr{ESD} باید هشدار، استیکر و اعلامیه بر روی زمین، سقف و دیوار وجود داشته باشد تا نشان دهند در این منطقه
بدون تجهیزات مناسب نمی‌توان حضور داشت.
\item
تجهیزاتی مانند صندلی، ابزارهای کار، قفسه، چرخ‌دستی، کفش، مچ‌بند، دستکش نیروهای این ناحیه باید \lr{ESDS}
\LTRfootnote{ESD Safe}
باشند و تجهیزات مناسب باید خریداری شوند. برای مثال یکی از تکنیک‌هایی که در قفسه‌ها استفاده می‌شود
این است که آنها را با مقاومت یک اهمی به زمین متصل می‌کنند.
تکنیک اتصال دستگاه‌ها و دیگر ابزارها به زمین، بسیار در \lr{EPA}ها مورد استفاده قرار می‌گیرد.
\item
پوشیدن لباس‌های مخصوص \lr{ESD} بسیار توصیه می‌شوند و حفاظی بین لباس‌های عادی و قطعات الکترونیکی می‌شوند تا از \lr{ESD}
جلوگیری کنند.
\item
جابه‌جا کردن تجهیزات به وسیله چرخ‌دستی بسیار توصیه می‌شود. زیرا چرخ‌دستی‌ها از اصطحکاک بیشتر جلوگیری می‌کنند.
\item
کمپانی‌ها باید تجهیزات در معرض خطر مانند تجهیزات \lr{ESDS} و یا \lr{PCB} را در پکیجی ایمن که مانند زندان فارادی
عمل می‌کند بسته بندی کنند و سپس آنها را منتقل کنند. روی بسته‌بندی آنها باید برچسب‌های هشدار \lr{ESD} وجود داشته باشد.
\item
مطابق استاندارد \lr{IEC 61340} کف زمین مدام باید با مواد شوینده خاصی شسته شود تا خاصیت عایق بودن خود را به درستی حفظ کند.
\end{itemize}
اقدامات امنیتی لازم بسیار فراتر از این چند مورد است. اما پس از انجام تمام آنها، \lr{EPA} نیازمند نگهداری و بررسی‌های متعدد می‌باشد.
مهندس ناظر، کنترل کیفیت و یا سیس ادمین شرایط تجهیزات و امنیت آنها را باید تضمین کند. میدان الکتریکی و مقاومت سطحی
تمام تجهیزات حاضر در \lr{EPA} از خود محصولات مانند سرورها و روترها تا ابزارهای محافظتی مانند صندلی، زمین، مچ‌بند و لباس‌های حفاظتی
باید مرتبا اندازه‌گیری و بررسی شود تا از ناحیه خطر عبور نکند. ابزارهای مورد نیاز استاتیک متر
\LTRfootnote{Electrostatic Field Meter}
و یا دستگاه اندازه‌گیری مقاومت سطحی
\LTRfootnote{Surface resistivity meter}
هستند.

برگرفته شده از سایت \lr{JP Anti Static}
\LTRfootnote{\url{http://jpantistatic.com/knowledge_esd_basic.html}}
فروشنده محصولات حفاظتی و مورد نیاز در \lr{EPA}ها.
\end{document}

